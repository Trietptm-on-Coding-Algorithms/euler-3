\documentclass[a4paper, 12pt]{article}
\usepackage{amsmath}
\begin{document}


\section{Problem}
"2520 is the smallest number that can be divided by each of the numbers from 
1 to 10 without any remainder.
What is the smallest positive number that is evenly divisible by all of the 
numbers from 1 to 20?"

\subsection{Restatement of Problem}
Find the least common multiple of the numbers 1-20.

\section{Solution}
No programming is needed to solve this problem.  I was trying to draft up 
an algorithm an accidentally stumbled acrossed the answer.

\begin{enumerate}
\item Perform the prime factorization on each number.
\item For each distinct prime, find the factor from the list creates in the
previous step with the largest power.
\item Multiplify these factors, this is the least common multiple.
\end{enumerate}

First, perform the prime factorization on each number.
\begin{align*} 
20 &= 2^2 * 5 \\
19 &= 19 \\
18 &= 2 * 3^2 \\
17 &= 17 \\
16 &= 2^4 \\
15 &= 3 * 5 \\
14 &= 2 * 7 \\
13 &= 13 \\
12 &= 2^2 * 3 \\
11 &= 11 \\
10 &= 2 * 5 \\
9 &= 3^2 \\
8 &= 2^3 \\
7 &= 7 \\
6 &= 2 * 3 \\
5 &= 5 \\
4 &= 2^2 \\
3 &= 3 \\
2 &= 2 
\end{align*}

Second, for each distinct prime number, find the factor with the largest power.
\begin{align*} 
20 &= 2^2 * \mathbf{5} \\
19 &= \mathbf{19} \\
18 &= 2 * \mathbf{3^2} \\
17 &= \mathbf{17} \\
16 &= \mathbf{2^4} \\
15 &= 3 * 5 \\
14 &= 2 * \mathbf{7} \\
13 &= \mathbf{13} \\
12 &= 2^2 * 3 \\
11 &= \mathbf{11} \\
10 &= 2 * 5 \\
9 &= 3^2 \\
8 &= 2^3 \\
7 &= 7 \\
6 &= 2 * 3 \\
5 &= 5 \\
4 &= 2^2 \\
3 &= 3 \\
2 &= 2 
\end{align*}

Third, multiply these numbers.
\begin{align*}
5 * 19 * 3^2 * 17 * 2^4 * 7 * 13 * 11 = 2^4 * 3^2 * 5 * 7 * 11 * 13 * 17 * 19 = 232792560
\end{align*}

\end{document}
